\Introduction
Работа с аудио-файлами сложного содержания, включающего не только оцифрованный звук, но и наборы инструкций для воспроизведения звука, требует постоянного обращения к их внутренней структуре, что может быть экономически и, с точки зрения производительности, дорогостоящей операцией. Чтобы избежать накладных расходов, которые может повлечь многократное использование сторонних приложений для работы с аудио-файлами, важно рассмотреть хранение таких файлов в удобном формате, подходящем под устанавливаемые требования использования.

Цель этой работы - разработать и реализовать метод распределенного хранения аудио-файлов в NoSQL-базе данных.

В рамках работы требуется решить следующие задачи.
\begin{enumerate}
    \item Рассмотреть способы представления звуковой информации и проанализировать существующие аудио-форматы.
    \item Проанализировать способы хранения аудио-файлов в различных базах данных.
    \item Разработать метод хранения аудио-файлов.
    \item Разработать программное обеспечение, реализующее разработанный метод.
    \item Исследовать зависимость времени работы метода от количества дорожек в аудио-файле.
\end{enumerate}
