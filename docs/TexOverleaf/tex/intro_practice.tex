\Introduction
Работа с аудио-файлами сложного содержания, включающего не только оцифрованный звук, но и наборы инструкций для воспроизведения звука, требует постоянного обращения к их внутренней структуре, что может быть экономически и, с точки зрения производительности, дорогостоящей операцией. Чтобы избежать накладных расходов, которые может повлечь многократное использование сторонних приложений для работы с аудио-файлами, важно рассмотреть хранение таких файлов в удобном формате, подходящем под устанавливаемые требования использования.

MongoDB - система управления базами данных, которая работает с документоориентированной моделью данных и для хранения данных использует JSON-подобный формат \cite{Mongo}.

Целью этой работы является выбор средств программной реализации метода
распределенного хранения аудио-файлов в NoSQL-базе данных, описание работы ПО и проведение исследования реализованного метода.

В рамках работы требуется решить следующие задачи.
\begin{enumerate}
\item Выбрать язык программирования.
\item Выбрать среду разработки.
\item Выбрать инструменты для замеров времени.
\item Описать инструкции сборки локального сервера MongoDB.
\item Описать структуру проекта.
\item Привести пример работы реализованного метода.
\item Провести исследование и проанализировать его результаты.
\end{enumerate}
