\chapter{ПРИЛОЖЕНИЕ А}
\parindent = 1.25cm
%Файл классификации:
%\lstinputlisting[firstline=4,lastline=79]{Classification.py}
%Файл нахождение дуги и аппроксимации:
%\lstinputlisting[firstline=171,lastline=289]{Learning.py}

\begin{lstlisting}[language=C, label=some-code, caption=Метод создания  документа MongoDB по спецификации MIDI (часть 1), label=lst:parser_midi1]
BSONObj parseMidiFile(const vector<char> buffer, const string fileName) {
    BSONObjBuilder builder;
    int i;
    int i_buff;

    // MThd ID
    char MThdId[5];
    for (i_buff = 0, i = 0; i_buff < 4; i_buff++, i++) {
        MThdId[i] = buffer[i_buff];
    }
    builder.append("ID", MThdId);

    builder.append("Name", fileName);

    // MThd Length
    char MThdLength[5];
    for (i_buff = 4, i = 0; i_buff < 8; i_buff++, i++) {
        MThdLength[i] = buffer[i_buff];
    }
    builder.appendBinData("Length", 4, BinDataGeneral, MThdLength);

    // MThd Format
    char MThdFormat[3];
    for (i_buff = 8, i = 0; i_buff < 10; i_buff++, i++) {
        MThdFormat[i] = buffer[i_buff];
    }
    builder.appendBinData("Format", 2, BinDataGeneral, MThdFormat);

    // MThd NumTracks
    char MThdNumTracks[3];
    for (i_buff = 10, i = 0; i_buff < 12; i_buff++, i++) {
        MThdNumTracks[i] = buffer[i_buff];
    }
    builder.appendBinData("NumTracks", 2, BinDataGeneral, MThdNumTracks);

    // MThd Division
    char MThdDivision[3];
    for (i_buff = 12, i = 0; i_buff < 14; i_buff++, i++) {
        MThdDivision[i] = buffer[i_buff];
    }
    builder.appendBinData("Division", 2, BinDataGeneral, MThdDivision);
\end{lstlisting}

\begin{lstlisting}[language=C, label=some-code, caption=Метод создания  документа MongoDB по спецификации MIDI (часть 2), label=lst:parser_midi2] 
    // MThd MTrks
    const int numTracks = convertByteToIntMthd(MThdNumTracks);
    BSONArray mtrks = BSONArray();
    int mtrkBegin = i_buff;
    int mtrkOffset = i_buff;
    int j;
    int i_mtrk;

    BSONObjBuilder mtrksBuilder(builder.subarrayStart("MTrks"));
    for (i_mtrk = 0; i_mtrk < numTracks; i_mtrk++) {
        BSONObjBuilder mtrkBuilder(mtrksBuilder.subobjStart("mtrk"));
        // MTrk ID
        char MTrkId[5];
        mtrkOffset += 4;
        for (j = mtrkBegin, i = 0; j < mtrkOffset; j++, i++) {
            MTrkId[i] = buffer[j];
        }
        mtrkBuilder.append("ID", MTrkId);

        mtrkBegin = mtrkOffset;
        mtrkOffset += 4;

        // MTrk Length
        char MTrkLength[5];
        for (j = mtrkBegin, i = 0; j < mtrkOffset; j++, i++) {
            MTrkLength[i] = buffer[j];
        }
        mtrkBuilder.appendBinData("Length", 4, BinDataGeneral, MTrkLength);
        const int mtrkLengthInt = convertByteToIntMtrk(MTrkLength);

        mtrkBegin = mtrkOffset;
        mtrkOffset += mtrkLengthInt;

        // MTrk Data
        char* Data = new char[mtrkLengthInt];
        for (j = mtrkBegin, i = 0; j < mtrkOffset; j++, i++) {
            Data[i] = buffer[j];
        }
        mtrkBuilder.appendBinData("Data", mtrkLengthInt, BinDataGeneral, Data);

        mtrkBegin = mtrkOffset;
\end{lstlisting}

\begin{lstlisting}[language=C, label=some-code, caption=Метод создания  документа MongoDB по спецификации MIDI (часть 3), label=lst:parser_midi3] 
        mtrkBuilder.done();
    }
    mtrksBuilder.done();
    BSONObj midi = builder.obj();

    return midi;
}
\end{lstlisting}

%%% Local Variables: 
%%% mode: latex
%%% TeX-master: "rpz"
%%% End: 
