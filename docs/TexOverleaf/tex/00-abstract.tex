% Также можно использовать \Referat, как в оригинале
\begin{abstract}

    Расчетно-пояснительная записка содержит \pageref{LastPage}\,страниц%
    \ifnum \totfig >0
    , \totfig~рисунков%
    \fi
    \ifnum \tottab >0
    , \tottab~таблиц%
    \fi
    %
    \ifnum \totbib >0
    , \totbib~источников%
    \fi
    %
    \ifnum \totapp >0
    , \totapp~приложений.%
    \else
    .%
    \fi

Ключевые слова: MIDI, NoSQL, MongoDB, BsonDocument, хранение.

Цель бакалаврской работы - разработка метода распределенного хранения аудио-файлов в NoSQL-базе данных.

Задачи, решаемые в работе:

\begin{enumerate}
    \item Рассмотреть способы представления звуковой информации и проанализировать существующие аудио-форматы.
    \item Проанализировать способы хранения аудио-файлов в различных базах данных.
    \item Разработать метод хранения аудио-файлов.
    \item Разработать программное обеспечение, реализующее разработанный метод.
    \item Исследовать зависимость времени работы метода от количества дорожек в аудио-файле.
\end{enumerate}

Область применения разрабатываемого метода - работа с аудио-файлами формата MIDI в музыкальной индустрии.

В данной выпускной бакалаврской работе проанализированы существующие аудио-форматы, выбран аудио-формат, дающий наиболее полную информацию о музыке, проанализированы существующие решения хранения такого формата в различных базах данных и выбран способ хранения, наиболее оптимальный для выбранного формата, проанализированы существующие модели данных и выбрана наиболее подходящая под выбранный способ хранения модель, а также выбрана база данных, которая ее использует. Разработано программное обеспечение, реализующее описанный метод. Произведено исследование времени работы метода и проанализированы полученные результаты.

%%% Local Variables: 
%%% mode: latex
%%% TeX-master: "rpz"
%%% End: 
\end{abstract} 